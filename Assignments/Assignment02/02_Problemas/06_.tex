\newpage
%%%%%%%%%%%%%%%%%%%%%%%%%%%%%%%%%%%%%%%%%%%%%%%%%%%%%%%%%%%%%%%%
%%%%%%%%%%%%%%%%%%%%%%%%%%%%%%%%%%%%%%%%%%%%%%%%%%%%%%%%%%%%%%%%
%%%%%%%%%%%%%%%%%%%%%%%%%% Enunciado %%%%%%%%%%%%%%%%%%%%%%%%%%%

\begin{myblock}
\phantomsection\addcontentsline{toc}{section}{Ejercicio \#9 | Modelo log-lineal para el dataset Titanic}
\section*{Ejercicio \#9 | Modelo log-lineal para el dataset Titanic}

Este problema analiza los datos históricos del hundimiento del Titanic en 1912. El objetivo es utilizar un \textbf{modelo log-lineal} para investigar las asociaciones e interacciones entre cuatro variables categóricas que describen a los pasajeros.

Los datos, disponibles en la librería \texttt{titanic} de R, se presentan en una tabla de contingencia de cuatro dimensiones con las siguientes variables:
\begin{itemize}
    \item \textbf{Class}: Clase del pasajero (1, 2, 3, Tripulación).
    \item \textbf{Sex}: Sexo (Male, Female).
    \item \textbf{Age}: Grupo de edad (Child, Adult).
    \item \textbf{Survived}: Estado de supervivencia (No, Yes).
\end{itemize}

Se debe ajustar un modelo log-lineal para evaluar la significancia de los siguientes efectos y así comprender la estructura de dependencias en los datos:
\begin{enumerate}
    \item \textbf{Efectos Principales}: Determinar si las frecuencias de pasajeros son uniformes a través de las categorías de \texttt{Class}, \texttt{Sex}, \texttt{Age} y \texttt{Survived}.
    \item \textbf{Interacciones de Dos Vías}: Evaluar la independencia entre cada par de variables (ej. \texttt{Sex} $\times$ \texttt{Survived}, \texttt{Class} $\times$ \texttt{Age}).
    \item \textbf{Interacciones de Orden Superior}: Investigar las dependencias más complejas, incluyendo todas las interacciones de tres vías y la interacción de cuatro vías.
\end{enumerate}

\end{myblock}

%%%%%%%%%%%%%%%%%%%%%%%%%%%%%%%%%%%%%%%%%%%%%%%%%%%%%%%%%%%%%%%%
%%%%%%%%%%%%%%%%%%%%%%%%%%%%%%%%%%%%%%%%%%%%%%%%%%%%%%%%%%%%%%%%

%%%%%%%%%%%%%%%%%%%%%%%%%%%%%%%%%%%%%%%%%%%%%%%%%%%%%%%%%%%%%%%%
%%%%%%%%%%%%%%%%%%%%%%%%%%%%%%%%%%%%%%%%%%%%%%%%%%%%%%%%%%%%%%%%

Para lograr investigar el comportamiento de independencia entre variables categóricas de estudio, implementamos
un modelo log-lineal jerárquico. Este enfoque permite evaluar de manera sistemática la contribución de interacciones
de creciente complejidad para explicar las frecuencias observadas en la tabla de contingencia. Se compararon cuatro
modelos anidados, cada uno representando una hipótesis específica sobre la relación entre las variables.

La selección del modelo óptimo se basó en el estadístico de razón de verosimilitud que compara el ajuste
de cad amoedlo propuesto con el ajuste del modelo saturado. Los modelos evaluados fueron:

\begin{itemize}
    \item \textbf{Modelo de Independencia Total (m0):} Este modelo base postula que no existe asociaci\'{o}n alguna entre ninguna de las variables. Matem\'{a}ticamente, se expresa como:
    $$ \log E[n_{ijkl}] = \mu + \alpha_i + \beta_j + \gamma_k + \delta_l $$
    La evaluaci\'{o}n de este modelo arroj\'{o} un ajuste extremadamente deficiente ($G^2 = 1243.66$ con 25 grados de libertad, $p \approx 0$). Este resultado permite rechazar de forma contundente la hip\'{o}tesis de que las variables son mutuamente independientes.

    \item \textbf{Modelo de Interacciones de Dos Vías (m2):} El segundo modelo incorpora todas las posibles asociaciones por pares entre las variables. Aunque represent\'{o} una mejora sustancial respecto al modelo de independencia, su ajuste segu\'{\i}a siendo estad\'{\i}sticamente inadecuado ($G^2 = 116.59$ con 13 grados de libertad, $p \approx 0$). Esto indica que las meras asociaciones entre pares de variables no son suficientes para capturar la complejidad de la estructura subyacente en los datos.

    \item \textbf{Modelo de Interacciones de Tres Vías (m3):} Este modelo representa el punto de inflexi\'{o}n en nuestro an\'{a}lisis. Al incluir todos los t\'{e}rminos de interacci\'{o}n de tres v\'{\i}as (y, por el principio de jerarqu\'{\i}a, los t\'{e}rminos de orden inferior), el modelo present\'{o} un ajuste excepcional a los datos observados ($G^2 = 2.73 \times 10^{-4}$ con 3 grados de libertad, $p \approx 0.999999$). Un valor de devianza pr\'{a}cticamente nulo y un p-valor cercano a 1 constituyen una fuerte evidencia de que este modelo explica casi la totalidad de la estructura de dependencia presente.

    \item \textbf{Modelo Saturado (m4):} Este modelo, que incluye la interacci\'{o}n de cuatro v\'{\i}as, reproduce perfectamente los datos por definici\'{o}n ($G^2 = 0$ con 0 grados de libertad). No aporta informaci\'{o}n explicativa adicional, pero sirve como referencia para confirmar que el modelo m3 ya ha capturado toda la estructura de dependencia relevante.
\end{itemize}

De ese modo, la evidencia estadística nos está diciendo que describir adecuadamente las relaciones enter variables
es indispensable incluir los términos de interacción de tres vías. El modelo $m3$ es la más precisa 
con la estructura de los datos. A su vez, la necesiadd de un término de interacción de tres vías implica
que la relación entre dos variables cualesquiera no es constante, sin oque stá moderada o condicionada 
por el valor ed una tercera variable. 

El efecto del Sexo sobre la Supervivencia no es el mismo en todas las Clases o para todos los grupos de Edad.
De manera an\'{a}loga, el impacto de la Edad en la probabilidad de Supervivencia se ve modificado por la Clase del individuo y su Sexo.
En esencia, la narrativa popular de ``mujeres y ni\~{n}os primero'' no fue un protocolo aplicado de manera uniforme. Su implementaci\'{o}n y, por tanto, la probabilidad de supervivencia, dependi\'{o} de manera crucial de la clase social a la que pertenec\'{i}an los individuos, revelando una interacci\'{o}n compleja y de orden superior.

Durante el an\'{a}lisis, se observ\'{o} que el estad\'{\i}stico chi-cuadrado de Pearson arrojaba un valor no definido (\texttt{NaN}). Esto se debe a la presencia de ceros estructurales en la tabla de contingencia (por ejemplo, la combinaci\'{o}n ``Tripulaci\'{o}n-Ni\~{n}o'' es l\'{o}gicamente imposible y su frecuencia observada es cero). En ciertos modelos, esto induce frecuencias esperadas ($E$) nulas para dichas celdas. Dado que el c\'{a}lculo del estad\'{\i}stico de Pearson implica una divisi\'{o}n por $E$ ($\sum \frac{(O-E)^2}{E}$), la presencia de un cero en el denominador provoca una indeterminaci\'{o}n matem\'{a}tica.

Por esta raz\'{o}n, el estad\'{\i}stico de raz\'{o}n de verosimilitud ($G^2$) es el m\'{e}todo preferido y m\'{a}s robusto para el an\'{a}lisis de tablas de contingencia que contienen ceros estructurales, ya que su c\'{a}lculo no presenta esta limitaci\'{o}n.

\clearpage